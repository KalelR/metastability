%%%%%%%%%%%%%%%%%%%%%%%%%%%%%%%%%%%%%%%%%%%%%%%%%%%%%%%%%%%%%%%%%%%%%%%%%%%%%%%%%%%%%%%%%%%%%%%%%%%%%%%%%%%%%%%%%%%%%%%%%%%%%%%%%%%%%%%%%%%%%%%%%%%%%%%%%%%
% This is just an example/guide for you to refer to when submitting manuscripts to Frontiers, it is not mandatory to use Frontiers .cls files nor frontiers.tex  %
% This will only generate the Manuscript, the final article will be typeset by Frontiers after acceptance.   
%                                              %
%                                                                                                                                                         %
% When submitting your files, remember to upload this *tex file, the pdf generated with it, the *bib file (if bibliography is not within the *tex) and all the figures.
%%%%%%%%%%%%%%%%%%%%%%%%%%%%%%%%%%%%%%%%%%%%%%%%%%%%%%%%%%%%%%%%%%%%%%%%%%%%%%%%%%%%%%%%%%%%%%%%%%%%%%%%%%%%%%%%%%%%%%%%%%%%%%%%%%%%%%%%%%%%%%%%%%%%%%%%%%%

%%% Version 3.4 Generated 2018/06/15 %%%
%%% You will need to have the following packages installed: datetime, fmtcount, etoolbox, fcprefix, which are normally inlcuded in WinEdt. %%%
%%% In http://www.ctan.org/ you can find the packages and how to install them, if necessary. %%%
%%%  NB logo1.jpg is required in the path in order to correctly compile front page header %%%

\documentclass[utf8]{frontiersSCNS} % for Science, Engineering and Humanities and Social Sciences articles
%\documentclass[utf8]{frontiersHLTH} % for Health articles
%\documentclass[utf8]{frontiersFPHY} % for Physics and Applied Mathematics and Statistics articles

%\setcitestyle{square} % for Physics and Applied Mathematics and Statistics articles
\usepackage{url,hyperref,lineno,microtype,subcaption}
\usepackage[onehalfspacing]{setspace}

\linenumbers


% Leave a blank line between paragraphs instead of using \\


\def\keyFont{\fontsize{8}{11}\helveticabold }
\def\firstAuthorLast{Rossi {et~al.}} %use et al only if is more than 1 author
\def\Authors{Kalel Luiz Rossi\,$^{1,*}$, Roberto Cesar Budzisnki\,$^{1}$, Bruno Rafael Reichert Boaretto \,$^{1}$, Sergio Roberto Lopes \,$^{1}$, Ulrike Feudel \,$^{2}$}
% Affiliations should be keyed to the author's name with superscript numbers and be listed as follows: Laboratory, Institute, Department, Organization, City, State abbreviation (USA, Canada, Australia), and Country (without detailed address information such as city zip codes or street names).
% If one of the authors has a change of address, list the new address below the correspondence details using a superscript symbol and use the same symbol to indicate the author in the author list.
% \def\Address{$^{1}${Department of Physics, Universidade Federal do Parana, 81531-980, Curitiba, PR, Brazil. \\
% $^{2}$ Theoretical Physics/Complex Systems, ICBM, Carl von Ossietzky University Oldenburg, 26111 Oldenburg, Germany}
% The Corresponding Author should be marked with an asterisk
% Provide the exact contact address (this time including street name and city zip code) and email of the corresponding author

\def\corrAuthor{Kalel Luiz Rossi}
\def\corrEmail{kalelluizrossi@gmail.com}




\begin{document}
\onecolumn
\firstpage{1}

\title[DiscussionMetastability]{Discussing Metastability in Neuroscience} 

\author[\firstAuthorLast ]{\Authors} %This field will be automatically populated
\address{} %This field will be automatically populated
% \correspondance{} %This field will be automatically populated

% \extraAuth{}% If there are more than 1 corresponding author, comment this line and uncomment the next one.
%\extraAuth{corresponding Author2 \\ Laboratory X2, Institute X2, Department X2, Organization X2, Street X2, City X2 , State XX2 (only USA, Canada and Australia), Zip Code2, X2 Country X2, email2@uni2.edu}


\maketitle


\begin{abstract}

%%% Leave the Abstract empty if your article does not require one, please see the Summary Table for full details.
\section{}


\tiny
 \keyFont{ \section{Keywords:} keyword, keyword, keyword, keyword, keyword, keyword, keyword, keyword} %All article types: you may provide up to 8 keywords; at least 5 are mandatory.
\end{abstract}

\section{Introduction}
Importance of metastability

Problem with definitions. Solution through mini-review, discussion, and proposal of a unifying definiiton.

How to quantify metastability, following the unifying definition?

Dependence on scales.



% \begin{enumerate}
    % \item Metastability is a common term in several scientific areas, with agreed upon definitions 
    % \item In neuroscience, several important works have used this term, with different meanings.
    % \item In the neuroscience literature, this term is also used. Important works highlight the importance of metastable behavior, for example as the dynamical basis for cognition
    % \item Despite the several papers and its imporatnce for brain functioning, metastability is rather loosely defined in the literature. Works provide definitions ranging from ... to ... . 
    % \item Despite all the individual definitions being reasonable and consistent in the works themselves, a term so common and important should be well-defined and a consensus among researchers in the field. This then avoids several possible problems, like confusing different behaviors that, unintentionally, have the same name. 
    % \item In this paper, we provide a mini-review regarding different conceptions of metastability in the literature. We propose that metastabiltiy can be defined as the incessant changes in ..., following \cite{friston_2000}. This encompasses the definitions in the literature, and calls attention to the fact that each operates on different scales and depends on different experimental apparatus.
    % \item Following \cite{betzel_2017}, we identify three relevant scales present in studies of complex networks. These are ... 
    % \item With this, we then discuss implications of such a definition, consider the meaning of a degree of metastability and examine a few ways to quantify it.
    % \item We then point out that the degree metastability can be largely different on different scales, or levels. This important points is rarely discussed in the literature, as far as we know. Works measuring metastability on a global level do not study this behavior on a local level, and may miss important behaviors. With this, we emphasize the importance of studying different scales and their interactions in complex system, like the brain. 
% \end{enumerate}



\section{Definitions of metastability}
Scanning the neuroscience literature, we extracted definitions of metastability used either explicitly or implicitly by authors. If multiple definitions could be identified, all of them were extracted. 


\subsection{Definition 1a - Variability of states }
\label{fundam:ssec:varstates}
Metastability here denotes the regime with a successive expression of the system's states over time.


A state can be concretely described as a set of observables characterizing or representing the system, like neuronal firing rates \cite{lacamera_2019}, a degree of phase synchronization \cite{alderson_2020, lee_2017, vasa_2015, hellyer_2014}, or the system's dynamical variables \cite{beimgraben_2019}. It can also be left as an abstract concept \cite{werner_2007}.


Since each of these states is successively replaced by another, none of them are equilibria. They are either transiently stable (were stable, but a change of parameters made them unstable), or are simply unstable states that are visited for some time ("attractor-like" \cite{vasa_2015, hellyer_2014}). In either case, they are generally called metastable states (metastates).

% NUMBER OF STATES DISCRETE?
\cite{lacamera_2019} requires that the transitions between states be abrupt, "jump-like". 



\subsection{Definition 1b - Variability of activity patterns}
Metastability here denotes the regime with a successive expression of activity patterns over time \cite{friston_1997, friston_2000, varela_2001}. Karl Friston requires these activity patterns to be "distinct, self-limiting and stereotyped" \cite{friston_1997}, referring to them as transients. 

**DESCRIBE FRISTONs LFP**

Activity of the system here means also a set of numbers describing, or reflecting the system's behavior. They can be the system's own variables, like membrane potentials, or local electric potentials \cite{roberts_2019}, or results of measurements, like local-field potential \cite{friston_1997, friston_2000}.

The patterns can be temporal or even spatial. In \cite{roberts_2019}, successive waves of electric potential are identified in whole-brain models, each denoting a spatial pattern, and their succession denotes metastability.

Each pattern can naturally reflect, or represent, the system's state, so that definitions 1a and 1b can be equivalent. 



\subsection{Definition 1c - Variability of synchronization or phase configurations}
\label{ssec:metastability-synchronization}
Metastability here refers directly to the variability in time of degrees of synchronization, or to oscillation phases, with no mention of states. It can denote a (i) variability of the global degree of phase synchronization \cite{cabral_2011, deco_2017}; (ii) variability of the states of phase configurations \cite{deco_2017}; (iii) variability of synchronization between different nodes \cite{deco_2017}; (iv) variability in the relative phases of nodes \cite{poncealvarez_2015}; (v) variability in the synchrony of each individual community in the network in time \cite{shanahan_2010, wildie_2012}. 

*** SEE IF SHANAHAN USES STATES***
All these cases can be viewed as a subset of definitions 1a or 1b, if the synchronization measure defines a system's state or activity pattern. 


\subsection{Definition 1d - Variability of regions in phase-space}
Metastability here refers to a regime with transitions between regions in phase space \cite{hudson_2017, beimgraben_2019, rabinovich_2008, cavanna_2018}. The trajectory of the system spends time in certain regions, and then moves to other regions.

***EXPLAIN OTHERS**
For example, \cite{rabinovich_2008} considers the specific case where a state is a saddle, and metastability occurs due to a heteroclinic cycle.
% colocar o que os outros casos dizem tambem?

Points in phase space are defined by the system's dynamical variables, which then represent its state \cite{cavanna_2018, beimgraben_2019}. Thus, this definition is a phase-space view of definition 1a. 


\subsection{Definition 1e - Variability of regions in energy landscape}
Metastability here refers to a regime with transitions between local minima of energy in an energy landscape. This is the definition in neuroscience closest to the one in physics. In this case, the system transitions from one state to another due to either external perturbations or to another dimension in the landscape \cite{gili_2018, cavanna_2018}. 

If the energy value describes a system's state, or represents its activity pattern, then this definition can be considered a specific case of 1a and 1b. 


\subsection{Definition 2 - Regime for integration and segregation of neural assemblies}
Metastability is often viewed as a dynamic regime that naturally implements the dual need for integration and segregation in the brain. The most common approach is to define metastability through one of the previous definitions, and consider integration-segregation as a consequence. However, \cite{fingelkurts_2001, fingelkurts_2004} define metastability directly as the regime with this tendency of integration-segregation. According to their theory of Operational Architectonics, this tendency produces the cognitive or behavioral processes in the brain and, therefore, metastability the regime behind them. These processes are constituted by a succession of different acts, each of which can is called a metastable state. 

\subsection{Discussion}
Do states, or activity patterns, have to characterize the system completely? uniquely? 

1a and 1b equivalent?

Should the separation of time-scales be explicity required? That is, the period between activity patterns?

Can jumps between metastates be continuous, or only abrupt?

Should a recurrence in the activity patterns be required?

Should metastability be endogenous (hellyer) aka spontaneous, or can it be forced?


\section{Quantifications of metastability}


\section{Metastability and brain scales}
\cite{tognoli_2014} fala sobre importancia do estudo de escalas e problema de inferencia

Def 1c:
On a topological scale, these definitions vary from a microscopic level (comparing nodes), to mesoscopic level (communities), to a macroscopic level (global). 

Discrete number of metastable states
Given a fixed time period, how many different states?

Continuous number of metastable states
?




% \section*{Conflict of Interest Statement}
% %All financial, commercial or other relationships that might be perceived by the academic community as representing a potential conflict of interest must be disclosed. If no such relationship exists, authors will be asked to confirm the following statement: 

% The authors declare that the research was conducted in the absence of any commercial or financial relationships that could be construed as a potential conflict of interest.

% \section*{Author Contributions}

% The Author Contributions section is mandatory for all articles, including articles by sole authors. If an appropriate statement is not provided on submission, a standard one will be inserted during the production process. The Author Contributions statement must describe the contributions of individual authors referred to by their initials and, in doing so, all authors agree to be accountable for the content of the work. Please see  \href{http://home.frontiersin.org/about/author-guidelines#AuthorandContributors}{here} for full authorship criteria.

% \section*{Funding}
% Details of all funding sources should be provided, including grant numbers if applicable. Please ensure to add all necessary funding information, as after publication this is no longer possible.

% \section*{Acknowledgments}
% This is a short text to acknowledge the contributions of specific colleagues, institutions, or agencies that aided the efforts of the authors.

% \section*{Supplemental Data}
%  \href{http://home.frontiersin.org/about/author-guidelines#SupplementaryMaterial}{Supplementary Material} should be uploaded separately on submission, if there are Supplementary Figures, please include the caption in the same file as the figure. LaTeX Supplementary Material templates can be found in the Frontiers LaTeX folder.

% \section*{Data Availability Statement}
% The datasets [GENERATED/ANALYZED] for this study can be found in the [NAME OF REPOSITORY] [LINK].
% Please see the availability of data guidelines for more information, at https://www.frontiersin.org/about/author-guidelines#AvailabilityofData

\bibliographystyle{frontiersinSCNS_ENG_HUMS} % for Science, Engineering and Humanities and Social Sciences articles, for Humanities and Social Sciences articles please include page numbers in the in-text citations
%\bibliographystyle{frontiersinHLTH&FPHY} % for Health, Physics and Mathematics articles
\bibliography{bibliography}

%%% Make sure to upload the bib file along with the tex file and PDF
%%% Please see the test.bib file for some examples of references

\section*{Figure captions}

%%% Please be aware that for original research articles we only permit a combined number of 15 figures and tables, one figure with multiple subfigures will count as only one figure.
%%% Use this if adding the figures directly in the mansucript, if so, please remember to also upload the files when submitting your article
%%% There is no need for adding the file termination, as long as you indicate where the file is saved. In the examples below the files (logo1.eps and logos.eps) are in the Frontiers LaTeX folder
%%% If using *.tif files convert them to .jpg or .png
%%%  NB logo1.eps is required in the path in order to correctly compile front page header %%%

\begin{figure}[h!]
\begin{center}
\includegraphics[width=10cm]{logo1}% This is a *.eps file
\end{center}
\caption{ Enter the caption for your figure here.  Repeat as  necessary for each of your figures}\label{fig:1}
\end{figure}


\begin{figure}[h!]
\begin{center}
\includegraphics[width=15cm]{logos}
\end{center}
\caption{This is a figure with sub figures, \textbf{(A)} is one logo, \textbf{(B)} is a different logo.}\label{fig:2}
\end{figure}

%%% If you are submitting a figure with subfigures please combine these into one image file with part labels integrated.
%%% If you don't add the figures in the LaTeX files, please upload them when submitting the article.
%%% Frontiers will add the figures at the end of the provisional pdf automatically
%%% The use of LaTeX coding to draw Diagrams/Figures/Structures should be avoided. They should be external callouts including graphics.

\end{document}
